\documentclass[12pt]{exam}
\usepackage[utf8]{inputenc}

\usepackage{graphicx}
\usepackage{makecell}
\usepackage{minibox}
\usepackage{multirow}
\usepackage{lastpage}
\usepackage{amsmath}
\usepackage[margin=0.5in]{geometry}
\usepackage{amssymb}
\usepackage{xcolor}
\usepackage{datetime}
\usepackage{amsmath}
\usepackage{mathtools}


\pagestyle{headandfoot}

\newdateformat{monthyeardate}{%
  \monthname[\THEMONTH], \THEYEAR}
%\newdateformat{\monthyeardate}{\monthname[\THEMONTH], \THEYEAR}




\newcommand{\paper}[5]{
    \setcounter{page}{1}
    
    \begin{minipage}{\textwidth}
    \Large{\textbf{Tougaloo College}}\\
    %\vspace{0.1in}\\  
   % Bachelor of the Science of Engineering \\
    %Academic Year #4\\
    %\vspace{0.1in}\\
    \textbf{#2}\\
    \textbf{#5 - #3}\\
    \end{minipage}
    \hfill
   % \begin{minipage}{3in}
      %  \centering
      %  \includegraphics[scale=0.23]{logo.jpg}
    %\end{minipage}\\
    %\hfill
   % \vspace{0.1in}\\
        %\begin{minipage}{6in}
         %   \textbf{\hspace{0.25in}Due Date : #1}
        %\end{minipage}
    
    \vspace{0.1in}
    \rule[1ex]{\textwidth}{2pt}
    }



\extrafootheight{.5in}
\cfoot{Page ~\thepage ~of ~\numpages}


\printanswers


\marginpointname{ \points}
\pointsinrightmargin
\pointpoints{ Point}{ Points}
\setlength{\rightpointsmargin}{3cm}
\pointsdroppedatright
\addpoints
%\qformat{\textbf{Question \thequestion} \hfill}

%%%%%%%%%%%%%%%%%%%%%%%%%%%%%%%%%%%%%%%%%%%%%%%%%%%%%%%%
%%%%%%%%%%%%%%%%%%%%%%%%%%%%%%%%%%%%%%%%%%%%%%%%%%%%%%%%


\begin{document}
%\paper{duration}{Course}{Semester}{academic year}{Examination}
\paper{}{MAT222 - Calculus II}{Spring, 2025}{}{Howework 09}

\section*{7.3 Exercises}
\begin{questions}



\question[] Evaluate the integral.

\begin{parts}
    \part[] \(\displaystyle \int \dfrac{x^2}{\sqrt{9-x^2}}~dx\)
    \begin{solution}
        \[\dfrac{9}{2} \left\{\sin^{-1}(x/3) - \dfrac{x\sqrt{9-x^2}}{9}\right\}+C\]
    \end{solution}
    \part[] \(\displaystyle \int_{0}^{1/2} x \sqrt{1-4x^2}~dx\)
    \begin{solution}
        \(\dfrac{1}{12}\)
    \end{solution}
\end{parts}





%%%%%%%%%%%%%%%%%%%%%%%%%%%%%%%%%%%%%%%%%%%%%%%%%%%%%%%%%
%%%%%%%%%%%%%%%%%%%%%%%%%%%%%%%%%%%%%%%%%%%%%%%%%%%%%%%%%
\end{questions}
\end{document}


