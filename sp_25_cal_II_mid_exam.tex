\documentclass[12pt]{exam}
\usepackage[utf8]{inputenc}

\usepackage{graphicx}
\usepackage{makecell}
\usepackage{minibox}
\usepackage{multirow}
\usepackage{lastpage}
\usepackage{amsmath}
\usepackage[margin=0.5in]{geometry}
\usepackage{amssymb}
\usepackage{xcolor}
\usepackage{datetime}
\usepackage{amsmath}
\usepackage{mathtools}


\pagestyle{headandfoot}

\newdateformat{monthyeardate}{%
  \monthname[\THEMONTH], \THEYEAR}
%\newdateformat{\monthyeardate}{\monthname[\THEMONTH], \THEYEAR}

\newcommand{\N}{\mathbb{N}}

\newcommand\dint{\displaystyle{\int}}
\newcommand{\paper}[5]{
    \setcounter{page}{1}
    
    \begin{minipage}{\textwidth}
    \Large{\textbf{Tougaloo College}}\\
    %\vspace{0.1in}\\  
   % Bachelor of the Science of Engineering \\
    %Academic Year #4\\
    %\vspace{0.1in}\\
    \textbf{#2}\\
    \textbf{#5 - #3}\\
    \end{minipage}
    \hfill
   % \begin{minipage}{3in}
      %  \centering
      %  \includegraphics[scale=0.23]{logo.jpg}
    %\end{minipage}\\
    %\hfill
   % \vspace{0.1in}\\
        \begin{minipage}{6in}
            \textbf{\hspace{0.25in}Duration : #1 }
        \end{minipage}
        \hfill
        \begin{minipage}{3in}
            \textbf{March 19, 2025}
        \end{minipage}
    
    \vspace{0.1in}
    \rule[1ex]{\textwidth}{2pt}
    }



\extrafootheight{.5in}
\cfoot{Page ~\thepage ~of ~\numpages}


%\printanswers


\marginpointname{ \points}
\pointsinrightmargin
\pointpoints{ Point}{ Points}
\setlength{\rightpointsmargin}{3cm}
\pointsdroppedatright
\addpoints
%\qformat{\textbf{Question \thequestion} \hfill}

%%%%%%%%%%%%%%%%%%%%%%%%%%%%%%%%%%%%%%%%%%%%%%%%%%%%%%%%
%%%%%%%%%%%%%%%%%%%%%%%%%%%%%%%%%%%%%%%%%%%%%%%%%%%%%%%%


\begin{document}
%\paper{duration}{Course}{Semester}{academic year}{Examination}
\paper{50 min}{MAT222 - Calculus II}{Spring, 2025}{}{MID-EXAM}

\noindent \textbf{Name :}  ...........................................................................................................\\
\vspace{0.1in}\\
\textbf{ID Number} : .................................................................\\
\vspace{0.2in}\\
\hspace*{0.2in}\textbf{Instructions to Candidates }
\begin{itemize}
    \item Calculators are \textbf{NOT} allowed. 
    \item This paper consists of $6$ questions.
    \item Answer all questions. 
    \item All questions carry marks as indicated for each question or part thereof.
    %\item Do not use red ink in answering the questions.
    \item All drawings or sketches, if any, should be produced clearly.
    \item Assume reasonable values for any data not given with the question paper. Clearly state any assumptions.
    %\item All answers should be provided in English language.
    %\item An electronic non-programmable calculator approved by the Faculty may be used.
    %\item All examinations are conducted in accordance with the Rules \& Regulations currently in force at the University.
\end{itemize}

\newpage
%\section*{Question 1}
\begin{questions}

\question[10] Use Midpoint Rule with the value \(n=5\) to approximate the integral (no need to simplify your answer):
\[
\int_0^1 \sqrt{x+1}~dx
\] 
\vfill
\droptotalpoints
\newpage

\question Evaluate the integrals:
\begin{parts}
    \part[10] \(\displaystyle
\int_0^1 (x^e+e^x)~dx
\)
\droppoints
\vspace{2.5in}

\part[10] \(\displaystyle
    \int_{-2}^1 \dfrac{1}{x^4}~dx
\) 
\droppoints
\vspace{2.5in}

\part[10] \(\displaystyle
\int 4x^3 e^{x^4}~dx
\)
\droppoints
\vfill
\newpage

\part[10] \(\displaystyle
\int_1^2 \dfrac{e^{1/x}}{x^2}~dx
\)
\droppoints
\vspace{2.5in}

\part[10] \(\displaystyle
\int \sin^3 \theta \cos^4 \theta~d\theta
\)
\droppoints
\vfill



\end{parts}
\droptotalpoints
\newpage

\question[20] Sketch the region enclosed by the given curves, then find the area of the region. \[
y= \sin x,~ y=x, ~x =\pi/2,~x=\pi.
\]
\vfill
\droptotalpoints
\newpage

\question[20] Use the washer or cylindrical shell method to find the volume of the solid obtained by rotating the region bounded by the curves $y^2=x$ and $x=2y$ about the $y-$axis. 

\vfill
\droptotalpoints
\newpage

\question[10] Find the average value of the following function on the interval \([-1,1]\). 
\[
f(x)=\dfrac{x^2}{(x^3+3)^2}
\] 
\vfill
\droptotalpoints
\newpage

\question[15] Evaluate the following integral using integration by parts.
\[
\int t^2 \sin {\beta t}~dt,
\] where \(\beta \) is a constant. 
\vfill
\droptotalpoints






%%%%%%%%%%%%%%%%%%%%%%%%%%%%%%%%%%%%%%%%%%%%%%%%%%%%%%%%%
%%%%%%%%%%%%%%%%%%%%%%%%%%%%%%%%%%%%%%%%%%%%%%%%%%%%%%%%%
\end{questions}
\end{document}


