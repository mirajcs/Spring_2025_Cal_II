\documentclass{beamer}

\usepackage{amsmath, amssymb}
\usepackage{tikz-cd}
\usepackage{xcolor}
\usepackage{graphicx}

\title{MAT222: Calculus II}
\author{\textbf{Miraj Samarakkody}}
\institute{Tougaloo College}
\date{04/02/2025}

\begin{document}

\begin{frame}
    \titlepage
\end{frame}





\begin{frame}{}
\begin{center}
    \Huge{Integration of Rational Functions by Partial Fractions}
\end{center}
\end{frame}

\begin{frame}{Motivation}
In this section, we show how to integrate any rational function by expressing it as a sum of simpler fractions, called \textit{partial fraction}, that we already know how to integrate. 
\end{frame}

    \begin{frame}{Example 1}
    Find \[\int \dfrac{x^3+x}{x-1}~dx\]\\ \pause
    \vspace{0.5in}
    Next we study some different cases. 
    \end{frame}

\begin{frame}{CASE I: The denominator \(Q(x)\) is a product of distinct linear factors.}    
We can write \[Q(x)=(a_1 x+b_1)(a_2x+b_2)\dots (a_k x +b_k),\] where no factor is repeated. \\ \pause
\vspace{0.2in}
In this case the partial fraction theorem states that there exist constants \(A_1,A_2,A_3, \dots , A_k\) such that \[\dfrac{R(x)}{Q(x)}= \dfrac{A_1}{a_1x+b_1}+ \dfrac{A_2}{a_2x+b_2}+ \dots + \dfrac{A_k}{a_k x +b_k}.\]
\end{frame}

\begin{frame}{Example 2}
Write the partial fraction decomposition of \[\dfrac{x^2 +2x -1}{2x^3+3x^2-2x}\]
\end{frame}

\begin{frame}{Example 2}
    Use and alternative method to write the partial fraction decomposition of \[\dfrac{x^2 +2x -1}{2x^3+3x^2-2x}\]
\end{frame}




\end{document}